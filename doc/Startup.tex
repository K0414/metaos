\documentclass[11pt,a4paper]{article}
\usepackage{graphicx}
\usepackage{theorem}
\usepackage{color}
\usepackage{listings}

\title{MetaOS startup with a Python file. Internal reference}
\author{Sergio Alvarez (setelena@gmail.com),
    Luis F. Canals (luisf.canals@gmail.com)}

\begin{document}

\date{\today}
\maketitle


\section{Abstract}

\section{Involved classes}

\lstset{language=Java,frame=single,tabsize=2,basicstyle=\tiny}
\lstinputlisting{../src/java/com/metaos/engine/Engine.java}

As it's seen from the code, a R source code named \emph{correlation.r}
where a class \emph{correlator} is defined. The class should have got the
methods \emph{memo} and \emph{show}, as in this example:

\lstset{language=R,frame=single,tabsize=2,basicstyle=\tiny}
\lstinputlisting{../src/R/correlation.r}



\section{Code explanation}

The following code (\emph{rintegration.py}) uses the same described principle 
in the previous section but letting the name of R source containing the 
class as a runtime parameter:

\lstset{language=Python,frame=single,tabsize=2,basicstyle=\tiny}
\lstinputlisting{../src/python/rintegration.py}

In this case, interface for R class has been modified, to create a simple
predictor with two methods, \emph{learn} and \emph{predict}. An example of
predictor based on linear regression might look like this:

From the example, we write down the interface all R-class should satisfy
to be compatible with \emph{rintegration.py} requirements (in pseudocode 
Java-like):

\lstset{language=Java,frame=single,tabsize=4,basicstyle=\tiny}
\begin{lstlisting}

    interface PredictorInR {
        // Learns a new relation x->y
        void learn(double x, double y);

        // Tries to predict the value of y from x
        double predict(double x);
    }

\end{lstlisting}
\label{PredictorInR}

\section{Other languages. R, for instance}


\section{Startup from graphical console}

The code in \emph{rintegration.py} can be invoked even from command line or
from visual interface. In both cases, file acting as a source of prices,
the main Python script (\emph{rintegration.py} in this case) and the R file
containing the source code of the class with the predictor following the 
interface described in the interface \ref{PredictorInR}.


\section{License terms}

This is a GNU Licensed Document. Modifications and copies of this document
must follow the GNU License, refering to authors and to the original document.
All other rights are reserved by the authors.
\end{document}
